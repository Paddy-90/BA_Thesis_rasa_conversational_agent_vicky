%\chapter*{Abstract}
\begin{abstract}                                 %abstract
Im Allgemeinen haben Lernende als Schüler, Studierende oder sich Fortbildene Motivationsprobleme 
beim Lernen. Es fehlt oft an der Kompetenz, selbständig und 
eigenmotiviert seinen Lernprozess zu gestalten.
Conversational Agents (CA) können als Ansprechpartner für die Gestaltung des Lernprozesses dienen.
Für einen optimalen Aufbau des Lernprozesses spielen charakteristische Merkmale wie der
Lernstil des Lernenden eine wichtige Rolle. Daher wurde in dieser Arbeit
die Machbarkeit analysiert, ob sowohl durch eine dialogbasierte Interaktion als auch durch 
ein Quiz-Spiel zwischen einem CA 
und einem Lernenden in natürlicher Sprache der Lernstil des Lernenden klassifiziert werden kann.
Für die Lernstilklassifikation wurde das Modell von Felder und Silverman (1988) verwendet.
Zudem wurde untersucht, inwiefern die Interaktion mit einem CA für den Lernenden motivierend erscheint.
Als Ergebnis dieser Arbeit wurde im Rahmen des Design Science Researchs ein Prototyp namens Vicky erstellt, welcher durch eine 
Studie (n=25) eruiert wurde. Damit wurden erste Tendenzen für eine mögliche Klassifikation des persönlichen 
Lernstils des Lernenden durch einen CA sowie einer positiven Auswirkung auf die Lernmotivation des Lernenden aufgezeigt.
Darüber hinaus werden weitere Forschungsmöglichkeiten zur Weiterentwicklung dargestellt.

  \textbf{Gender-Disclaimer:} \\
  Die in dieser Arbeit durchgängige verwendete männliche From bezieht sich 
  immer zugleich auf alle Geschlechter. Dieses Vorgehen wurde aufgrund der leichten 
  Lesbarkeit gewählt.
\end{abstract}